\chapter {Linear Algebra} 
Den Første side er bare noter...
Noget om transformationer og standard matricen.

\begin{align*}
    \textbf{x}=
    \begin{pmatrix}x_1 
        \\x_2 \\ x_3 
    \end{pmatrix} &= 
    \begin{pmatrix}
        x_1 \\ 0 \\ 0  
    \end{pmatrix} + 
    \begin{pmatrix}
        0 \\ x_2 \\ 0  
    \end{pmatrix} + 
    \begin{pmatrix}
        0 \\ 0 \\ x_3  
    \end{pmatrix}\\ &= x_1  
   \begin{pmatrix}
        1 \\ 0\\ 0  
   \end{pmatrix}+ x_2 
   \begin{pmatrix}
        0 \\ 1 \\ 0  
   \end{pmatrix} + x_3 
   \begin{pmatrix}
        0 \\ 0 \\ 1  
   \end{pmatrix} \\ &= x_1 \textbf{e}_1 + x_2\textbf{e}_2+x_3 \textbf{e}_3
\end{align*}

\begin{align*}
f(\textbf{x})&=f(x_1 \textbf{e}_1 + x_2\textbf{e}_2+x_3 \textbf{e}_3)\\ 
&= f(x_1 \textbf{e}_1) + f(x_2 \textbf{e}_2) + f(x_3 \textbf{e}_3) \\
&=x_1 f(\textbf{e}_1) + x_2 f(\textbf{e}_2) + x_3 f(\textbf{e}_3)
\\ &= [f(\textbf{e}_1) f(\textbf{e}_2 ) f(\textbf{e}_3)] 
\begin{pmatrix}
    x_1 \\x_2 \\ x_3 
\end{pmatrix} &= A \textbf{x}
\end{align*}
f er en matrix afbildning

Nogle eksempler jeg har skrevet ned i mit hæfte, da vi løste opgaver:

Here is an example on how to show that a vector is an eigenvector of a matrix $A$ and how to determine what the corresponding eigenvalue is.... 

\begin{align*}
A = \begin{bmatrix}-5 & -4 \\8 & 7 \end{bmatrix}, \textbf{v} =\begin{bmatrix}1 \\-2\end{bmatrix}\\
\begin{bmatrix}-5 & -4 \\8 & 7 \end{bmatrix} \begin{bmatrix}1 \\-2\end{bmatrix}\\ = \begin{bmatrix}-5 + 8 \\8  -14 \end{bmatrix} = \begin{bmatrix}3 \\-6 \end{bmatrix} = 3\begin{bmatrix}1\\-2 \end{bmatrix}
\end{align*}

(Her skal være en intro; Koncept med lineær algebra, hvorfor vi går i gang med det, og hvorfor vi dækker det LiAl som vi gør)

\begin{definition}{Principles of Linearity}
A transform $f: \mathbb{R}^n \rightarrow \mathbb{R}^m $ is linear if the following is true... 
\begin{align*}
f(u+v)&=f(u)+f(v)\\
f(c u)&=c f u ,  \text{where c is a constant. }
\end{align*}

Matrix-vector product...
\begin{align*}
A(\textbf{u}+\textbf{v})=& A\textbf{u}+A\textbf{v}\\
A(c \textbf{u})=& c f\textbf{u}
\end{align*}
With means that all matrix transforms are linear..
måske se side 171
\end{definition}

A matrix is a rectangular array of scalars. The size of a matrix is determined by its number of rows $m$ and number of columns $n$. 
For instance an $A_{ij}$ matrix, where the scalar in the $i$-th row and $j$-th column is called the (i, j)-entry of the $A_{ij}$ matrix:
\begin{align*}
    A_{ij} = 
    \begin{bmatrix}
    a_{1 1} & a_{1 2} & \cdots & a_{1 j}\\
    a_{2 1} & a_{2 2} & \cdots & a_{2 j}\\
    \vdots  &   \vdots &  \ddots   & \vdots \\
    a_{i 1} & a_{i 2} & \cdots & a_{i j}\\
    \end{bmatrix}
\end{align*}
\cite[4]{LiAl}
To determine the solution set of a system of linear equations or whether the system is inconsistent, there are some elementary row operations that can be done on a matrix $A$, which makes it possible to replace it with a equivalent system of linear equations, meaning a system of linear equations with the same solution, but more easily solved.
\begin{definition}{Elementary Row Operations}
There are three elementary row operations, that can be used to determine an equivalent system of linear equations, for a given system of linear equations.
\begin{enumerate}
    \item Interchange Operation:
    It is allowed to let two rows switch place 
    Notation: $A\xrightarrow{\textbf{r}_i\leftrightarrow \textbf{r}_t} B$ where $i$ and $t$ are rows
    \item Scaling Operation:
    Multiplying any row of a matrix with the same nonzero scalar.
    Notation: $A\xrightarrow{c\textbf{r}_i\rightarrow \textbf{r}_i} B$
    \item Row Addition Operation:
    Add a multiple of one row of the matrix to another row
    Notation: $A\xrightarrow{c\textbf{r}_i+\textbf{r}_t\rightarrow \textbf{r}_t} B$
\end{enumerate}
\cite[32]{LiAl}
\end {definition}
An example with a given system of linear equations, where the  elementary row operations will be used to replace the given system with an equivalent system of linear equations, which will be more easily solved.
\begin{example}{Elementary Row Operations}
 Having the given system of linear equations
\begin{align*}
    x_1-x_2-2x_4-8x_5&=-3\\
    -2x_1+x_3+2x_4+9x_5 &= 5\\
    3x-1-2x_3-3x_4-15x_5&=-9.
\end{align*}
The augmented matrix can be written as

\begin{align*}
[A \textbf{b}] =
	&\begin{bmatrix}
	1 & 0 & -1 &-2 & -8 & -3 \\
	-2 & 0 & 1 & 2 & 9 & 5 \\
	3 &0 & -2 & -3 & -15 & -9 
   \end{bmatrix} \\
  \xrightarrow{\substack{r_2+2r_1\rightarrow r_2\\  r_3-3r_1\rightarrow r_3}}
  &\begin{bmatrix}
 	 1 & 0 & -1 &-2 & -8 & -3 \\
 	 0 & 0 & -1 & -2 & -7 & -1 \\
	 0 &0 & 1 & 3 & 9 & 0
  \end{bmatrix}\\
  \xrightarrow{r_3+r_2\rightarrow r_3}
    &\begin{bmatrix}
  	 1 & 0 & -1 &-2 & -8 & -3 \\
 	 0 & 0 & -1 & -2 & -7 & -1 \\
	 0 &0 & 0 & 1 & 2 & -1
       \end{bmatrix}.
\end{align*}
\label{exa:rowoperations}
\end{example}
When replacing a given system of linear equations with one, which are more easily solved, the most simple equivalent system of linear equations is called the reduced row echelon form. 
\begin{definition}{Reduced Row Echelon Form}
A matrix in row echelon form is satisfying the following three conditions
\begin{enumerate}
    \item Each nonzero row lies above every zero row
    \item The leading entry of a nonzero row is in a column to the right of the column containing the leading entry of any preceding row
    \item All entries below a leading entry are $0$
\end{enumerate}
For a matrix to be in reduced echelon form it has to satisfy two conditions more. these are
\begin{enumerate}
    \item For a column with a leading entry, all other entries have to be 0
    \item The leading entries has to be equal to 1
\end{enumerate}
\cite[33]{LiAl}
\end{definition}
To determine whether a system of linear equations is consistent or inconsistent, the concepts pivot positions and pivot columns
becomes useful. Using the conditions a row echelon form is satisfying, the number of pivot columns is equal to the number of leading entries in the given matrix, and a pivot position is the position of the leading entry.  
\begin{example}{Reduced Row Echelon Form}
It can be observed from the matrix in Example \ref{exa:rowoperations} that there is three pivot positions, and thereby three pivot columns.
\begin{align}
     \begin{bmatrix}
  	 \circled{1} & 0 & -1 &-2 & -8 & -3 \\
 	 0 & 0 & \circled{-1} & -2 & -7 & -1 \\
	 0 &0 & 0 & \circled{1} & 2 & -1
       \end{bmatrix}
\label{examplepivot}
\end{align}
There are no pivot position in the last column of the augmented matrix, hence the system is consistent. All entries in column two is equal to zero, hence the system has infinite solutions. To determine the most simple equivalent system of the given linear equations, the reduced row echelon form is calculated:
\begin{align*}
  \xrightarrow{\substack{r_1+2r_3\rightarrow r_1\\r_2+2r_3\rightarrow r_2}}
    &\begin{bmatrix}
  	    1 & 0 & -1 &0 & -4 & -5 \\
 	    0 & 0 & -1 & 0 & -3 & -3 \\
	    0 &0 & 0 & 1 & 2 & -1
     \end{bmatrix}\\
  \xrightarrow{\substack{r_1+r_2\rightarrow r_1\\-1r_2\rightarrow r_2}}
  &\begin{bmatrix}
        1 & 0 & 0 &0 & -1 & -2 \\
 	    0 & 0 & 1 & 0 & 3 & 3 \\
	    0 &0 & 0 & 1 & 2 & -1
     \end{bmatrix}
    \end{align*}
(En eller anden tekst som overgang)

	\begin{align*}
		x-x_5 =-2   &\Rightarrow x_1=x_5 -2\\
        x_3 + 3x_5 =3 &\Rightarrow x_3=-3x_5+3\\
        x_4 +2x_5 = -1 &\Rightarrow	x_4=-2x_5-1
	\end{align*}
    \begin{align*}
    	\textbf{x}=
        \begin{bmatrix}
       	    x_1\\ x_2 \\ x_3\\ x_4 \\x_5
        \end{bmatrix} =
        \begin{bmatrix}
        	x_5- 2\\ x_2\\ -3x_5 +3\\-2x_5-1\\ x_5 
        \end{bmatrix}=
        \begin{bmatrix}
      	    -2\\ 0 \\ 3 \\-1 \\0
        \end{bmatrix} + x_2
        \begin{bmatrix}
            0 \\ 1 \\ 0 \\ 0 \\ 0 
        \end{bmatrix} + x_5
        \begin{bmatrix}
            1 \\ 0 \\ -3 \\ -2 \\ 1
        \end{bmatrix}
    \end{align*}
\end{example}

\begin{definition}{Rank and Nullity of a Matrix}
The rank of a $m x n$ matrix $A$, is defined as the number of pivot columns, or the number of nonzero rows of the reduced echelon form of $A$. The rank is denoted as rank$(A)$
The nullity is defined as the opposite of rank, $n - $rank. Nullity is denoted as nullity$(A)$.
\cite[47]{LiAl}
\end{definition}

\begin{definition} {The Span of a Set of Vectors}
A linear combinations of a nonempty set of vectors $\mathcal{S} = \{ \textbf{u}_1,\textbf{u}_2 \cdots \textbf{u}_k\}$ is denoted as $c_1 \textbf{u}_1 + c_2\textbf{u}_2+ \cdots+ c_k\textbf{u}_k$, where $\textbf{u}_1,\textbf{u}_2 \cdots \textbf{u}_k$ are in $\mathbb{R}^n$ and $c_1, c_2 \cdots c_k$ are scalars.
The span of the set of vectors is a set of all linear combinations in $\mathbb{R}^n$
\cite[66]{LiAl}
\end{definition}

vise et kort ekemspel på span og rank - udgangspunkt i det tidligere eksempel. 
\eqref{examplepivot} - ref til matricen med pivot.

\section{Subspace, Basis and Dimension}
(En bette intro)

\begin{definition}{Subspace}
A subspace is a set of vectors $W$ in $\mathbb{R}^n$ which have the following three properties. 

\begin{enumerate}
    \item The zero vector belongs to $W$\\
    \textbf{Closure properties:}
    \item The sum of any pair of vectors that belongs to $W$, belongs to $W$ too. This means $W$ is closed under vector addition.
    \item Every scalar multiple of a vector that belongs to $W$, belongs to $W$ too. This means $W$ is closed under scalar multiplication.
\end{enumerate}
\cite[227]{LiAl}
\label{exa:SubspaceDef}
\end{definition}
\begin{example}{Determine whether $W$ is a subspace or not}
\begin{align*}
    W = \begin{Bmatrix}
    \begin{bmatrix}
    w_1\\w_2\\w_3\\w_4
    \end{bmatrix}
        \in \mathbb{R}^4: 2w_1 + 5w_2 - 7w_3 -2w_4 = 0 
    \end{Bmatrix}
\end{align*}
\begin{enumerate}
    \item To determine whenever the first property from\ref{exa:SubspaceDef} is satisfied, it is calculated that
    $2(0) + 5(0) - 7(0) -2(0) = 0 $ Hence $\textbf{w}=\textbf{0}$
    \item Having two vectors $\textbf{u}=\begin{bmatrix}
    u_1\\u_2\\u_3\\u_4\end{bmatrix}$ and $\textbf{v}=\begin{bmatrix}
    v_1\\v_2\\v_3\\v_4\end{bmatrix}$ in $W$ it must satisfy that $\textbf{u}+\textbf{v} = \begin{bmatrix}
    u_1 + v_1 \\ u_2 + v_2 \\ u_3 + v_3
    \end{bmatrix}$\\
    Doing the calculations yields
    \begin{align*}
        &\,2(u_1+v_1)+5(u_2+v_2)-7(u_3+v_3)-2(u_4+v_4)\\
        =& (2u_1+5u_2-7u_3+2u_4)+(2v_1+5v_2-7v_3+2v_4)\\ =& 0 + 0 \\=& 0
    \end{align*}
    \item Having a the vector \textbf{u} it most satisfy c
\end{enumerate}
\end{example}

\begin{theorem}{More about span}
The span of a definite nonempty subset of $\mathbb{R}^n$ is a subspace of $\mathbb{R}^n$.
\cite[231]{LiAl}
\end{theorem}
Proooff????$\leftarrow$

\begin{example}{Span as a subspace}
\begin{align*}
    W = \begin{Bmatrix}
    \begin{bmatrix}
    2s-5t\\3r+s-2t\\r-4s+3t\\-r+2s
    \end{bmatrix}
        \in \mathbb{R}^4:\text{r, s and t are scalars}
    \end{Bmatrix}
\end{align*}
Written on parameterised vector form
\begin{align*}
    W =
    r \begin{bmatrix}
        0 \\ 3 \\1 \\ -1
    \end{bmatrix}
    +s\begin{bmatrix}
        2 \\ 1 \\ -4 \\ 2
    \end{bmatrix}
    +t\begin{bmatrix}
        -5 \\ -2 \\ 3 \\ 0
    \end{bmatrix}
\end{align*}
\begin{align*}
    \mathcal{S}=
    \begin{Bmatrix}
    \begin{bmatrix}
        0 \\ 3 \\1 \\ -1
    \end{bmatrix}, 
    \begin{bmatrix}
        2 \\ 1 \\ -4 \\ 2
    \end{bmatrix},
    \begin{bmatrix}
        -5 \\ -2 \\ 3 \\ 0
    \end{bmatrix}
    \end{Bmatrix}
\end{align*}
Hence $\mathcal{S}$ is a subspace of $\mathbb{R}^4$
\end{example}

\begin{definition}{Basis}

\end{definition}
