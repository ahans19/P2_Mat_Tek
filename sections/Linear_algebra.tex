\chapter {Linear Algebra} 
Noget om transformationer og standard matricen.

\begin{align*}
    \textbf{x}=
    \begin{pmatrix}x_1 
        \\x_2 \\ x_3 
    \end{pmatrix} &= 
    \begin{pmatrix}
        x_1 \\ 0 \\ 0  
    \end{pmatrix} + 
    \begin{pmatrix}
        0 \\ x_2 \\ 0  
    \end{pmatrix} + 
    \begin{pmatrix}
        0 \\ 0 \\ x_3  
    \end{pmatrix}\\ &= x_1  
   \begin{pmatrix}
        1 \\ 0\\ 0  
   \end{pmatrix}+ x_2 
   \begin{pmatrix}
        0 \\ 1 \\ 0  
   \end{pmatrix} + x_3 
   \begin{pmatrix}
        0 \\ 0 \\ 1  
   \end{pmatrix} \\ &= x_1 \textbf{e}_1 + x_2\textbf{e}_2+x_3 \textbf{e}_3
\end{align*}

\begin{align*}
f(\textbf{x})&=f(x_1 \textbf{e}_1 + x_2\textbf{e}_2+x_3 \textbf{e}_3)\\ 
&= f(x_1 \textbf{e}_1) + f(x_2 \textbf{e}_2) + f(x_3 \textbf{e}_3) \\
&=x_1 f(\textbf{e}_1) + x_2 f(\textbf{e}_2) + x_3 f(\textbf{e}_3)
\\ &= [f(\textbf{e}_1) f(\textbf{e}_2 ) f(\textbf{e}_3)] 
\begin{pmatrix}
    x_1 \\x_2 \\ x_3 
\end{pmatrix} &= A \textbf{x}
\end{align*}
f er en matrix afbildning

Nogle eksempler jeg har skrevet ned i mit hæfte, da vi løste opgaver:

Here is an example on how to show that a vector is an eigenvector of a matrix $A$ and how to determine what the corresponding eigenvalue is.... 

\begin{align*}
A = \begin{bmatrix}-5 & -4 \\8 & 7 \end{bmatrix}, \textbf{v} =\begin{bmatrix}1 \\-2\end{bmatrix}\\
\begin{bmatrix}-5 & -4 \\8 & 7 \end{bmatrix} \begin{bmatrix}1 \\-2\end{bmatrix}\\ = \begin{bmatrix}-5 + 8 \\8  -14 \end{bmatrix} = \begin{bmatrix}3 \\-6 \end{bmatrix} = 3\begin{bmatrix}1\\-2 \end{bmatrix}
\end{align*}

(Her skal være en intro; Koncept med lineær algebra, hvorfor vi går i gang med det, og hvorfor vi dækker det LiAl som vi gør)

\begin{definition}{Linearity}
$f(x)=Ax$

A transform $f: \mathbb{R}^n \rightarrow \mathbb{R}^m $ is linear if the following is true... 
\begin{align*}
f(u+v)=f(u)+f(v)\\
f(c u)=c f u ,  \text{where c is a constant. }
\end{align*}

Matrix-vector produkt...
\begin{align*}
A(\textbf{u}+\textbf{v})=A\textbf{u}+A\textbf{v}\\
A(c \textbf{u})=c f\textbf{u}
\end{align*}
With means that all matrix transforms are linear..
måske se side 171
\end{definition}

\begin{definition}{Matrix}
A matrix is a rectangular array of scalars. The size of a matrix is determined by its number of rows $m$ and number of columns $n$. 
For instance a $A_{ij}$ matrix, where the scalar in the $i$-th row and $j$-th column is called the (i, j)-entry of the $A_{ij}$ matrix:
\begin{align*}
    A_{ij} = 
    \begin{bmatrix}
    a_{1 1} & a_{1 2} & \cdots & a_{1 j}\\
    a_{2 1} & a_{2 2} & \cdots & a_{2 j}\\
    \vdots  &   \vdots &  \ddots   & \vdots \\
    a_{i 1} & a_{i 2} & \cdots & a_{i j}\\
    \end{bmatrix}
\end{align*}
\end{definition}
To determine the solution set of a system of linear equations or whether the system is inconsistent, there are some elementary row operations that can be done on a matrix $A$, which makes it possible to replace it with a equivalent system of linear equations, meaning a system of linear equations with the same solution, but more easily solved.
\begin{definition}{Elementary Row Operations}
There are three elementary row operations, that can be used to determine an equivalent system of linear equations, for a given system of linear equations. 
\begin{enumerate}
    \item Interchange Operation:
    It is allowed to let two rows switch place 
    Notation: $A\xrightarrow{\textbf{r}\leftrightarrow \textbf{r}} B$ 
    \item Scaling Operation:
    Multiplying any row of a matrix with the same nonzero scalar.
    Notation: $A\xrightarrow{c\textbf{r}\rightarrow \textbf{r}} B$
    \item Row Addition Operation:
    Add a multiple of one row of the matrix to another row
    Notation: $A\xrightarrow{c\textbf{r}+\textbf{r}\rightarrow \textbf{r}} B$
\end{enumerate}
\end {definition}

\begin{example}{Elementary Row Operations}
This is an example with a given system of linear equations, where the  elementary row operations will be used to replace the given system with an equivalent system of linear equations, which will be more easily solved. Having the given system of linear equations
\begin{align*}
    x_1-x_2-2x_4-8x_5&=-3\\
    -2x_1+x_3+2x_4+9x_5 &= 5\\
    3x-1-2x_3-3x_4-15x_5&=-9
\end{align*}
the augmented matrix can be written as
\begin{align*}
[A \textbf{b}] =
	&\begin{bmatrix}
	1 & 0 & -1 &-2 & -8 & -3 \\
	-2 & 0 & 1 & 2 & 9 & 5 \\
	3 &0 & -2 & -3 & -15 & -9 
   \end{bmatrix} \\
  \xrightarrow{\substack{r_2+2r_1\rightarrow r_2\\  r_3-3r_1\rightarrow r_3}}
  &\begin{bmatrix}
 	 1 & 0 & -1 &-2 & -8 & -3 \\
 	 0 & 0 & -1 & -2 & -7 & -1 \\
	 0 &0 & 1 & 3 & 9 & 0
  \end{bmatrix}\\
  \xrightarrow{r_3+r_2\rightarrow r_3}
    &\begin{bmatrix}
  	 1 & 0 & -1 &-2 & -8 & -3 \\
 	 0 & 0 & -1 & -2 & -7 & -1 \\
	 0 &0 & 0 & 1 & 2 & -1
       \end{bmatrix}
\end{align*}
\label{exa:rowoperations}
\end{example}
When replacing a given system of linear equations with one, which are more easily solved, the most simple equivalent system of linear is called the reduced row echelon form. 
To define how to determine the reduced row echelon form it is necessary to state concepts as zero row, nonzero row and a leading entry. Zero row is a row where all the entries are 0 and nonzero row otherwise. (Leading entry ved ikke lige hvordan det nemmest forklares)
\begin{definition}{Reduced Row Echelon Form}
A matrix in row echelon form is satisfying the following three conditions
\begin{enumerate}
    \item Each nonzero row lies above every zero row
    \item The leading entry of a nonzero row is in a column to the right of the column containing the leading entry of any preceding row
    \item All entries below a leading entry are $0$
\end{enumerate}
For a matrix to be in reduced echelon form it has to satisfy two conditions more. these are
\begin{enumerate}
    \item For a column with a leading entry, all other entries have to be 0
    \item The leading entry has to be 1
\end{enumerate}
\end{definition}

\begin{example}{Row Echelon Form}
    In Example \autoref{exa:rowoperations} there were determined an equivalent system of linear equations, which is more easily solved. 
\end{example}

\begin{example}{Reduced Row Echelon Form}
\begin{align*}
  \xrightarrow{\substack{r_1+2r_3\rightarrow r_1\\r_2+2r_3\rightarrow r_2}}
    &\begin{bmatrix}
  	    1 & 0 & -1 &0 & -4 & -5 \\
 	    0 & 0 & -1 & 0 & -3 & -3 \\
	    0 &0 & 0 & 1 & 2 & -1
     \end{bmatrix}\\
  \xrightarrow{\substack{r_1+r_2\rightarrow r_1\\-1r_2\rightarrow r_2}}
  &\begin{bmatrix}
        1 & 0 & 0 &0 & -1 & -2 \\
 	    0 & 0 & 1 & 0 & 3 & 3 \\
	    0 &0 & 0 & 1 & 2 & -1
     \end{bmatrix}
\end{align*}
\end{example}

\begin{example}{Parameterised Vectorform }
	\begin{align*}
		x-x_5 =-2   &\Rightarrow x_1=x_5 -2\\
        x_3 + 3x_5 =3 &\Rightarrow x_3=-3x_5+3\\
        x_4 +2x_5 = -1 &\Rightarrow	x_4=-2x_5-1
	\end{align*}
	En eller anden kommentar om, at $x_2$ og $x_5$ er frie variabler. 
    \begin{align*}
    	\textbf{x}=
        \begin{bmatrix}
       	    x_1\\ x_2 \\ x_3\\ x_4 \\x_5
        \end{bmatrix} =
        \begin{bmatrix}
        	x_5- 2\\ x_2\\ -3x_5 +3\\-2x_5-1\\ x_5 
        \end{bmatrix}=
        \begin{bmatrix}
      	    -2\\ 0 \\ 3 \\-1 \\0
        \end{bmatrix} + x_2
        \begin{bmatrix}
            0 \\ 1 \\ 0 \\ 0 \\ 0 
        \end{bmatrix} + x_5
        \begin{bmatrix}
            1 \\ 0 \\ -3 \\ -2 \\ 1
        \end{bmatrix}
    \end{align*}
\end{example}

\begin{definition} {The Span of a Set of Vectors}
A linear combination is a vector of the form $c_1 \textbf{u}_1 + c_2\textbf{u}_2+ \cdots+ c_k\textbf{u}_k$, where $\textbf{u}_1,\textbf{u}_2 \cdots \textbf{u}_k$ are in $\mathbb{R}^n$ and $c_1, c_2 \cdots c_k$ are scalars. 
\end{definition}

\begin{definition}{Rank and Nullity of a Matrix}

\end{definition}