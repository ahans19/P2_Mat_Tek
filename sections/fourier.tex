\chapter{The Discrete Fourier Transform}
In this chapter the discrete Fourier transform (DFT) is defined and from this the fast Fourier transform is derived.\\

In this project, the DFT is used to transform an audio signal into the different frequencies in which it is composed of.

\begin{definition}{The Discrete Fourier Transform}
    The (DFT) transforms a discrete signal $f$ in the time domain with $N$ terms in to a discrete sequence $F$ in the frequency domain also with $N$ terms, defined as 
    \begin{align*}
        F[k]=\sum^{N-1}_{n=0}f[n]e^{-2jnk\pi/N}.
    \end{align*}
\end{definition}
The outcome of 
The DFT can be linearly transformed by the following theorem. 
\begin{theorem}{The Discrete Fourier Transform as a matrix}
    By making the definition 
    \begin{align*}
        W_N=e^{-2j\pi/N}
    \end{align*}
    then the discrete Fourier transform can be written as
    \begin{align*}
        F[k]=\sum^{N-1}_{n=0}f[n]W_N^{nk}.
    \end{align*}
    From this the $N\times N$ discrete Fourier transform matrix is defined as 
    \begin{align*}
        DFT_N=
        \begin{bmatrix}
             W_N^{0\cdot0} & W_N^{0\cdot1} & W_N^{0\cdot2} & \hdots & W_N^{0(N-1)} \\
             W_N^{1\cdot0} & W_N^{1\cdot1} & W_N^{1\cdot2} & \hdots & W_N^{1(N-1)} \\
             W_N^{2\cdot 0} & W_N^{2\cdot1} & W_N^{2\cdot2} & \hdots & W_N^{2(N-1)} \\
             \vdots & \vdots & \vdots & \ddots & \vdots \\
             W_N^{(N-1)0} & W_N^{(N-1)1} & W_N^{(N-1)2} & \hdots & W_N^{(N-1)(N-1)} \\
         \end{bmatrix}.
    \end{align*}
    Then the DFT of $\textbf{x}$ is expressed as the transforms
    \begin{align*}
        \textbf{y}=DFT_N\textbf{x}.
    \end{align*}
\end{theorem}
Since the DFT can be expressed as a matrix vektor product by THEORM ??? the DFT is a linear transformation.

\section{The Fast Fourier Transform}
The fast Fourier transform (FFT) is an method of reducing the number of computations needed to calculate the DFT. This is done by reducing the DFT matrix to a product of sparse matrices. 
To do so the following theorem is need
\begin{theorem}{The DFT when $N$ is even}
    Let $\mathbf{y}=DFT_{N}\mathbf{x}$ be the $N$-point DFT of $\mathbf{x}$ where $N$ is even and let $D_{N/2}$ be an $(N/2)\times(N/2)$ diagonal matrix with entries $(D_{N/2})_{n,n}=W_N^n$ for $0\leq n< N/2$. Then
    \begin{align*}
         \mathbf{y_1}&=DFT_{N/2}\mathbf{x}^{(e)}+D_{N/2}DFT_{N/2}\mathbf{x}^{(o)}\\ 
         \mathbf{y_2}&=DFT_{N/2}\mathbf{x}^{(e)}-D_{N/2}DFT_{N/2}\mathbf{x}^{(o)}.\\
    \end{align*}
    Where $\mathbf{y_1}$ is the first half $N/2$ entries of $\mathbf{y}$ and $\mathbf{y_2}$ is the second half set of entries. Hence $\mathbf{y_1},\mathbf{y_2}\in \mathds{C}^{N/2}$ and $\mathbf{x}^{(e)}$ and $\mathbf{x}^{(o)}$ are the even and odd entries of $\mathbf{x}$ respectively. Hence $\mathbf{x}^{(e)},\mathbf{x}^{(o)}\in \mathds{R}^{N/2}$.
\end{theorem}
\begin{proof}{The DFT when $N$ is even}
By the definition of the DFT,
    \begin{align*}
         F[k]&=\sum^{N-1}_{n=0}f[n]W_N^{nk}.\\
        \intertext{The sum of the DFT can be split into two sums, the even and the odd values of $k$ respectively.}\\
        &=\sum^{N/2-1}_{r=0}f[2r]W_N^{2rk}+\sum^{N/2-1}_{r=0}f[2r+1]W_N^{(2r+1)k}\\
        &=\sum^{N/2-1}_{r=0}f[2r](W_N^2)^{rk}+W_N^k\sum^{N/2-1}_{r=0}f[2r+1](W_N^2)^{rk}\\
        W_N^2 &= \exp\left(\frac{-j2(2\pi)}{N}\right) = \exp\left(\frac{-j2\pi}{N/2}\right) = W_{N/2}\\
        F[k] &= \sum^{(N/2)-1}_{r=0}f[2r]W^{rk}_{N/2}+W^k_N\sum^{(N/2)-1}_{r=0}f[2r+1]W^{rk}_{N/2}\\
    \end{align*}
     This can be written using the definition of $D_{N/2}$ where $0 \leq k < N/2$
     \begin{align*}
        \mathbf{y_1}=DFT_{N/2}\mathbf{x}^{(e)}+D_{N/2}DFT_{N/2}\mathbf{x}^{(o)}.\\ 
     \end{align*}
    If $k$ is limited to $0 \leq k < N/2$, then the second half of the DFT can be processed using the same methods as in the first half.
    \begin{align*}
        F[k+N/2] &= \sum^{N-1}_{n=0}f[n]W_N^{n(k+N/2)}\\
        &=\sum^{N/2-1}_{r=0}f[2r]W_N^{2r(k+N/2)}+\sum^{N/2-1}_{r=0}f[2r+1]W_N^{(2r+1)(k+N/2)}\\
        &=\sum^{N/2-1}_{r=0}f[2r](W_N^2)^{r(k+N/2)}+W_N^{k+N/2}\sum^{N/2-1}_{r=0}f[2r+1](W_N^2)^{r(k+N/2)}\\
        W_N^{N/2}&=\exp\left(\frac{-j2\pi}{N}\frac{N}{2}\right)=\exp\left(-j\pi\right)=-1\\
        F[k+N/2] &=\sum^{N/2-1}_{r=0}f[2r]W_{N/2}^{r(k+N/2)}-W_N^{k}\sum^{N/2-1}_{r=0}f[2r+1]W_{N/2}^{r(k+N/2)}.\\
    \end{align*}
    
    \begin{align*}
        \mathbf{y_2}=DFT_{N/2}\mathbf{x}^{(e)}-D_{N/2}DFT_{N/2}\mathbf{x}^{(o)}\\ 
     \end{align*}
\end{proof}

\begin{align*}
    \mathbf{y}=
    \begin{bmatrix}
        \mathbf{y_1}\\
        \mathbf{y_2}
    \end{bmatrix}
    =
    \begin{bmatrix}
        DFT_{N/2} & D_{N/2}DFT_{N/2} \\
        DFT_{N/2} & -D_{N/2}DFT_{N/2}
    \end{bmatrix}
    \begin{bmatrix}
        \mathbf{x}^{(e)} \\
        \mathbf{x}^{(o)}
    \end{bmatrix}
    =
    \begin{bmatrix}
        I_{N/2} & D_{N/2} \\
        I_{N/2} & -D_{N/2}
    \end{bmatrix}
    \begin{bmatrix}
        DFT_{N/2} & \mathbf{0} \\
        \mathbf{0} & DFT_{N/2}
    \end{bmatrix}
    \begin{bmatrix}
        \mathbf{x}^{(e)} \\
        \mathbf{x}^{(o)}
    \end{bmatrix}
\end{align*}
