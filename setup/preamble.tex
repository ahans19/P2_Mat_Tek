% Dokument setup, her må helst ikke pilles for meget, med mindre man ved hvad man laver!!!! Huskeregel, lad være med at slette noget før du har tjekket om de nye koder virker. Her kan tilføjes pakker efter ønske. 

\documentclass[hidelinks,11pt,twoside,a4paper,openright]{report}
\usepackage[utf8]{inputenc}
\usepackage[english]{babel}
% Use the palatino font
\usepackage[sc]{mathpazo}
\linespread{1.05}         % Palatino needs more leading (space between lines)
% Choose the font encoding
\usepackage[T1]{fontenc}

%%%%%%%%%%%%%%%%%%%%%%%%%%%%%%%%%%%%%%%%%%%%%%%%
% Graphics and Tables
% http://en.wikibooks.org/wiki/LaTeX/Importing_Graphics
% http://en.wikibooks.org/wiki/LaTeX/Tables
% http://en.wikibooks.org/wiki/LaTeX/Colors
%%%%%%%%%%%%%%%%%%%%%%%%%%%%%%%%%%%%%%%%%%%%%%%%
% load a colour package
\usepackage{xcolor}
\usepackage{tcolorbox}
\definecolor{aaublue}{RGB}{33,26,82}% dark blue
% The standard graphics inclusion package
\usepackage{graphicx}
% Set up how figure and table captions are displayed
\usepackage{float}
\usepackage{caption}

\captionsetup{%
  font=footnotesize,% set font size to footnotesize
  labelfont=bf % bold label (e.g., Figure 3.2) font
}
% Make the standard latex tables look so much better
\usepackage{array,booktabs}
% Enable the use of frames around, e.g., theorems
% The framed package is used in the example environment
\usepackage{framed}

%\Pdf pakke til bilag
\usepackage{pdfpages}

%%%%%%%%%%%%%%%%%%%%%%%%%%%%%%%%%%%%%%%%%%%%%%%%
% Mathematics
% http://en.wikibooks.org/wiki/LaTeX/Mathematics
%%%%%%%%%%%%%%%%%%%%%%%%%%%%%%%%%%%%%%%%%%%%%%%%
\usepackage{amsmath}
% Adds new math symbols
\usepackage{amssymb}
\usepackage[framed,amsmath,thmmarks]{ntheorem}

%%%%%%%%%%%%%%%%%%%%%%%%%%%%%%%%%%%%%%%%%%%%%%%%
% Page Layout
% http://en.wikibooks.org/wiki/LaTeX/Page_Layout
%%%%%%%%%%%%%%%%%%%%%%%%%%%%%%%%%%%%%%%%%%%%%%%%
% Change margins, papersize, etc of the document
\usepackage[
  inner=28mm,% left margin on an odd page
  outer=41mm,% right margin on an odd page
  ]{geometry}
  
  
% Modify how \chapter, \section, etc. look
% The titlesec package is very configureable
% Kapitel-titelformat
\usepackage{titlesec}
\titleformat{\chapter}[block]{\normalfont\Huge\bfseries}{\thechapter}{1em}{\Huge}
\titlespacing*{\chapter}{0pt}{-19pt}{0pt}

\titleformat*{\section}{\normalfont\Large\bfseries}
\titleformat*{\subsection}{\normalfont\large\bfseries}
\titleformat*{\subsubsection}{\normalfont\normalsize\bfseries}
\titleformat*{\paragraph}{\normalfont\normalsize\bfseries}
\titleformat*{\subparagraph}{\normalfont\normalsize\bfseries}


% Clear empty pages between chapters
\let\origdoublepage\cleardoublepage
\newcommand{\clearemptydoublepage}{%
  \clearpage
  {\pagestyle{empty}\origdoublepage}%
}
\let\cleardoublepage\clearemptydoublepage

% Change the headers and footers
\usepackage{fancyhdr}
\pagestyle{fancy}
\fancyhf{} %delete everything
\renewcommand{\headrulewidth}{0pt} %remove the horizontal line in the header
\fancyhead[RE]{\small\nouppercase\leftmark} %even page - chapter title
\fancyhead[LO]{\small\nouppercase\rightmark} %uneven page - section title
\fancyhead[LE,RO]{\thepage} %page number on all pages
% Do not stretch the content of a page. Instead,
% insert white space at the bottom of the page
\raggedbottom
% Enable arithmetics with length. Useful when
% typesetting the layout.
\usepackage{calc}

%%%%%%%%%%%%%%%%%%%%%%%%%%%%%%%%%%%%%%%%%%%%%%%%
% Bibliography
% http://en.wikibooks.org/wiki/LaTeX/Bibliography_Management
%%%%%%%%%%%%%%%%%%%%%%%%%%%%%%%%%%%%%%%%%%%%%%%%
\usepackage[backend=bibtex,
  bibencoding=utf8, sorting=none
  ]{biblatex}
\addbibresource{bib/mybib}

%%%%%%%%%%%%%%%%%%%%%%%%%%%%%%%%%%%%%%%%%%%%%%%%
% Misc
%%%%%%%%%%%%%%%%%%%%%%%%%%%%%%%%%%%%%%%%%%%%%%%%
% Add bibliography and index to the table of
% contents
\usepackage[nottoc]{tocbibind}
\sloppy
% Add the command \pageref{LastPage} which refers to the
% page number of the last page
\usepackage{lastpage}
% Add todo notes in the margin of the document
\usepackage[
%  disable, %turn off 
  colorinlistoftodos, %enable a coloured square in the list of todos
  textwidth=\marginparwidth, %set the width of the todonotes
  textsize=scriptsize, %size of the text in the todonotes
  ]{todonotes}

% Vores packages:
\usepackage{dsfont}
\usepackage{tikz}
\usetikzlibrary{shapes, shapes.geometric, arrows}
\usepackage{caption}
\usepackage{subcaption}
\usepackage{pgfplots}
\pgfplotsset{compat=1.14}
%\usepackage[parfill]{parskip}
\usepackage{multicol}
\usepackage{mathrsfs}
\usepackage{csquotes}
\usepackage[super]{nth}

%En test
\usepackage{pgf,tikz,pgfplots}
\pgfplotsset{compat=1.15}
\usepackage{mathrsfs}
\usetikzlibrary{arrows}


%%%%%%%%%%%%%%%%%%%%%%%%%%%%%%%%%%%%%%%%%%%%%%%%
% Hyperlinks
% http://en.wikibooks.org/wiki/LaTeX/Hyperlinks
%%%%%%%%%%%%%%%%%%%%%%%%%%%%%%%%%%%%%%%%%%%%%%%%
% Enable hyperlinks and insert info into the pdf
% file. Hypperref should be loaded as one of the 
% last packages
\usepackage{hyperref}
\usepackage{xcolor}
\hypersetup{%
    colorlinks,
	plainpages=false,%
	pdfauthor={Author(s)},%
	pdftitle={Title},%
	pdfsubject={Subject},%
	bookmarksnumbered=true,%
	colorlinks=false,%
	citecolor={blue!50!black},%
	filecolor=black,%
	linkcolor={red!50!black},% you should probably change this to black before printing
	urlcolor={blue!80!black},%
	pdfstartview=FitH%
	}
	
\usepackage{siunitx}